\graphicspath{{content/2_design/figures/}}
\section{Digital Motor Controller: Low-Side Switch}

\subsection{Configuration}

The low-side switch will be a FQD13N06L NMOS transistor driven by the ESP microcontroller. This transistor was chosen due to its
low turn-on voltage, eliminating the need for any gain circuitry. A resistor will be used in series from the ESP to the MOSFET to
prevent surge currents due to the capacitive nature of the MOSFET gate. A pull-down resistor will also be used to ensure
the circuit is fully off when the ESP pin is in a high impedance e.g. when the circuit is powered down or during startup.

The current sensor circuit will also be a non-inverting amplifier with a filter, similar to the circuit used for the analog wheel,
however care needs to be taken to ensure the PWM signal is filtered adequately. The sense resistor will be placed on the low side
of the MOSFET to avoid the use of a differential amplifier. Although this will cause the transistor's source to be slightly
above ground, the 

\subsection{Input Resistors}

In order to calculate $R_a$, the series input resistor from the ESP, the assumption is made that the resistance before the equivalent gate capacitance is zero.
Since the current equation for an RC circuit is $I(t) = I_0 e^{-t/\tau}$, with $I_0 = \frac{V}{R}$, the surge current will be $I_0$ at $t = 0$. Choose $I_0 = \SI{10}{mA}$
$\therefore R_a = \frac{V_{cc}}{I_0} = \frac{\SI{3.3}{V}}{\SI{10}{mA}} = \SI{330}{\ohm}$.

The pull-down resistor should be chosen so that the ESP is not loaded during operation, thereby affecting the voltage. If it is chosen that this resistor consumes
maximum additional current of $\SI{100}{\micro\ampere}$, the resistor should then be $R_{b(min)} = \frac{\SI{3.3}{V}}{\SI{100}{\micro\ampere}} = \SI{33}{\kilo\ohm}$.
Choose $R_b = \SI{100}{\kilo\ohm}$.

\subsection{Transistor Requirements}

It should be checked that the FQD13N06L transistor is suitable with regards to turn-on voltage, current capabilities, and series resistance.
\begin{itemize}
    \item According to \cite{datasheetFQD13N06L}, the maximum turn-on voltage of the transistor is 2.5 V. Since the ESP will output a 3.3 V PWM signal on its digital pins,
    the transistor will be turned on with the threshold passed.
    \item A maximum stall current of 750 mA is required to avoid damaging the transistor due to excess heat dissipation. Choose $I_{max} = \SI{1}{A}$ for additional headroom.
          At this maximum, with $V_{gs} = \SI{3}{V} \approx \SI{3.3}{V}$, $V_{ds} = $
\end{itemize}

\subsection{Noise Requirements}

