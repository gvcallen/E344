\graphicspath{{content/2_design/figures/}}
\section{Motor Controller}

\subsection{Configuration}

The motor controller circuit will consist of both a subtractor stage and a power stage. The subtractor will be a differential op-amp which will receive
the range sensor and DAC outputs as inputs. The output of this stage will feed the power stage with the "motor control" command. The power stage will use a common-collector circuit
with a TIP31C NPN transistor to provide current gain from the control command. A second transistor, the 2N3904, will be used to create a Darlington pair configuration to increase the input impedance.
A "flyback" diode will be connected from ground to the power stage's output to prevent sparking.

\subsection{Power Stage}{\label{motorController_design_powerStage}}

Since the load is a motor, current will always be sourced by the transistor (not sunk), and therefore the load can be connected directly to the
emitter without any bias resistor or AC coupling capacitor. Since the subtractor stage can provide a DC bias, there is also no need for a resistor input bias network.

According to the specifications, the output voltage range across the motor/load should be at least $\SI{0.5}{V} < V_L < \SI{6.2}{V}$,
however a higher maximum voltage is desireable. The TIP31C saturates at $V_{be} \approx \SI{0.7}{V}$ for $I_c = \SI{10}{mA}$ and at $V_{be} \approx \SI{0.95}{V}$ for $I_c = \SI{1}{A}$
(around the motor stall current) \cite{datasheetTIP31C}. Therefore, to ensure the motor is off, $V_{control} < \SI{0.5}{V} + \SI{0.7}{V} = \SI{1.2}{V}$,
and to power the motor fully, $V_{control} > \SI{6.2}{V} + \SI{0.95}{V} = \SI{7.15}{V}$. A minimum input range of $\SI{1.2}{V} < V_{control} < \SI{7.15}{V}$ is therefore chosen,
however the lower/upper limits may be decreased/increased respectively during subtractor design. With $I_{c(max)} \approx \SI{1}{A}$, the source needs to be capable of providing
$I_{s(max)} = \frac{\SI{1}{A}}{\beta} = \SI{40}{mA}$ since $\beta \approx 10^{1.4} = 25$ \cite{datasheetTIP31C}.


\subsection{Subtractor Stage}
The input voltage ranges into this stage are $\SI{0.2}{V} < V_{range} < \SI{3.3}{V}$ for the ultrasonic sensor, and $\SI{0.1}{V} < V_{speed} < \SI{3.1}{V}$ for the DAC,
with the output range being equal to the power stage's input range as discussed in Section \ref{motorController_design_powerStage}. Since a high $V_{speed}$ corresponds to
a low output voltage, the DAC input will be connected to the inverting terminal. No offset needs to be added at the non-inverting terminal
as the range sensor provides a constant 3.3 V when no object is near. A limit of $\SI{400}{\micro\ampere}$ should be kept for this circuit, excluding quiescent current.
Referring to the final circuit diagram in Figure below and according to \cite{opAmpSumDiff}, the equation for $V_{control}$ is given by:
\begin{equation}{\label{eqn:opAmpSumDiff}}
    % V_{control} = \left( V_{range} \cdot \frac{R_a}{R_a + R_b} \right) \left(1 + \frac{R_f}{R_s} \right) - V_{speed} \cdot \frac{R_f}{R_s}
    V_{control} = \left(V_{ref} \cdot R_r + V_{range} \cdot R_o \right) \left(\frac{1 + \frac{R_f}{R_s}}{Rr + Ro} \right) - V_{speed} \cdot \frac{R_f}{R_s}    
\end{equation}

\noindent Equation \ref{eqn:opAmpSumDiff} ultimately has 3 unknowns i.e. 3 degrees of freedom.

\noindent Component values can now be calculated:
\begin{itemize}
    \item Maximum current will flow through the feedback loop when $V_{control} = \SI{7.2}{V}$ and $V_{speed} = \SI{0}{V}$, with a value of $I_{max} = \frac{\SI{7.2}{V}}{R_s + R_f}$.
          To keep feedback current $< \SI{200}{\micro\ampere}$, $(R_s + R_f)_{min} = \frac{\SI{7.2}{V}}{\SI{200}{\micro\ampere}} = \SI{36}{\kilo\ohm}$.
    \item When no object close, the output voltage should be $< \SI{1.2}{V}$ for $V_{speed} = \SI{3.1}{V}$ and $> \SI{7.15}{V}$ for $V_{speed} = \SI{0.1}{V}$. This results
          in a minimum gain of $\frac{\SI{7.15}{V} - \SI{1.2}{V}}{\SI{3.1}{V} - \SI{0.1}{V}} = \SI{1.98}{V/V}$. Choose a gain of 2 i.e. $\frac{R_f}{R_s} = 2$.
          Choose $R_f = \SI{47}{\kilo\ohm}$ and $R_s = \SI{23.5}{\kilo\ohm} = \SI{18}{\kilo\ohm} + \SI{10}{\kilo\ohm}$pot.
    \item Now, $V_{control} = 3 \cdot V_{range} \cdot \frac{R_a}{R_a + R_b} - 2 \cdot V_{speed}$, with a single DAC step resulting in $\approx \SI{400}{mV}$ step.
          For $V_{speed} = \SI{0.1}{V}$ and $V_{range} = \SI{3.3}{V}$, $V_{control} \approx \SI{7.2}{V}$ therefore solve for $\frac{R_a}{R_a + R_b} = 0.747$.
\end{itemize}