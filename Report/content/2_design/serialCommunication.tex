\graphicspath{{content/2_design/figures/}}
\section{Serial Communication}

A serial communication protocol and procedure will be implemented that will allow control of the car, as well as telemetry readings.
The protocol will be implemented both over UART and Bluetooth to allow both USB and wireless control. The protocol will set out
the rules for communication for a "client" and a "server", including the types of messages and the format of the data.
The MCU and PC will then implement this protocol as server and client respectively.

\subsection{Protocol}

Table \ref{tab:protocolMessages} shows the list of messages that may either be sent by the client (a "request"),
or sent by the server (a "response"). Each message consists of an ID followed by a potential payload.
IDs 000 to 099 are reserved for "get" messages, and 100 to 199 are reserved for "set" messages.
\textit{Payload} refers to the data that will be sent be either the client \textit{or} the server.
\textit{void} implies that either the client \textit{or} the server must NOT pass any payload for that message.
\textit{N/A} implies that either the client \textit{or} the server must NEVER send that message.

\begin{table}[!htb]
  \centering
  \renewcommand{\arraystretch}{1.2}
  \begin{tabular}{ |c|c|c|c| }
    \hline
    \textbf{Description}         & \textbf{ID}        & \textbf{Payload (from client)}      & \textbf{Payload (from server)}    \\
    \hline
    Get All Information          & [000]              & void                                & [ID][Payload] for each "get"      \\
    \hline
    Get Left Wheel Current       & [001]              & void                                & [float32] (mA)                    \\
    \hline
    Get Right Wheel Current      & [002]              & void                                & [float32] (mA)                    \\
    \hline
    Get Left Sensor Range        & [003]              & void                                & [float32] (mm)                    \\
    \hline
    Get Right Sensor Range       & [004]              & void                                & [float32] (mm)                    \\
    \hline
    Get Battery Voltage          & [005]              & void                                & [float32] (mV)                    \\
    \hline
    Get Left Wheel Speed         & [006]              & void                                & [uint8]                           \\
    \hline
    Get Right Wheel Speed        & [007]              & void                                & [uint8]                           \\
    \hline
    Set Left Wheel Speed         & [100]              & [uint8]                             & N/A                               \\
    \hline
    Set Right Wheel Speed        & [101]              & [uint8]                             & N/A                               \\
    \hline
  \end{tabular}
  \caption{}
  \label{tab:protocolMessages}
\end{table}