\chapter{Detail design}\label{ch:detail_design}
%**********************************************
This process details the design process and calculations for the current sensor/amplifier circuit. It provides insight into
what decisions were made and the justification behind these decisions.

\section{Current sensor}\label{sec:current_sensor_design}
\subsubsection{Configuration}\label{sec:current_sensor_config}

The simple non-inverting amplifier configuration was decided on for this design. Although the differential amplifier provides explicit noise-rejection,
its filter design would prove rather complex and may require a second buffer stage compared to the single additional capacitor for the non-inverting case.
Overall, with proper filter design, the non-inverting amplifier should still ensure that there is little noise in the output signal.

The only addition to the researched configuration is that of an input resistor into the non-inverting input of the amplifier, followed by a capacitor
from the non-inverting input to ground. This RC combination determines the high cut-off of the filter, which can be used to block out the noise.

\subsubsection{Circuit diagram}\label{sec:current_sensor_circuit}
The following figure details the circuit design for the sensor and amplifier combination.

\begin{figure}[h!]
  \centering
  \includegraphics[width=.8\linewidth]{Figures/Circuit}
  \captionof{figure}{Circuit diagram of final configuration}  
  \label{fig:circuit-diagram}
\end{figure}

As can be see, a single +5 V and 0 V supply will be used. Although there are benefits to a dual rail supply, that would prove impractical
in the context of the larger system.



\pagebreak
Since input current of the circuit needs to be limited below 150 uA, high resistance values should be chosen:
\begin{itemize}
  \item Assuming $i_n = 0$, choose $\frac{V_{out(max)}}{(R_1 + R_2)} << 150 uA \therefore R_1 + R_2 >> \SI{20}{\kilo\ohm}$. Choose $R_1 = \SI{100}{\kilo\ohm}$.
  \item $G = 1 + \frac{R_2}{R_1}$
  \item 
\end{itemize}